\section{Quinta lezione: Foreste, alberi e cammini}
\subsection{Definizioni}
Una \emph{foresta} è un grafo senza cicli (aciclico). \\
Un \emph{albero} è una foresta connessa. \\

\noindent
Se F(V,E) è una foresta, allora $\mid E\mid = \mid V\mid-  n$ (con n numero delle componenti connesse di F) \\
Se T(V,E) è un albero, allora $\mid E\mid = \mid V\mid -1$. \\

\noindent
Se T(V,E) è un albero con almeno 2 nodi, allora T contiene almeno 2 nodi di grado 1. \\
\paragraph{Teorema} Sia T(V,E) un grafo connesso. Allora le seguenti affermazioni sono equivalenti:
\begin{itemize}
\item T non ha cicli.
\item Per ogni coppia di vertici distinti, x e y, esiste un unico cammino che li collega.
\item Il grafo T è minimalmente connesso, cioè la rimozione di un qualsiasi arco lo disconnette,
cioè ogni arco e' un taglio.
\end{itemize}
\textbf{DImostrazione:} \\
\(a \Rightarrow c \Rightarrow b \Rightarrow a \) \\

\noindent
\(a \Rightarrow c\) ) T connesso, senza cicli $\rightarrow$ ogni arco è un taglio. Suppongo che esista un arco $u,v$ tale che $T \setminus uv$ sia connesso. In $T \setminus uv$ esiste un cammino $P\ped{uv}$ che collega $u$ e $v$.\\
% cose a caso indecifrabili
$c \Rightarrow b$ ) T connesso tale che ogni arco è un taglio. \(\forall x, y \in V , x \neq y\) esiste un unico $P\ped{uv}$. \\
Suponiamo che esistano $x, y \in V, x \neq y$ tali che ci siano 2 cammini differenti che li collegano ($P' \ped{xy}$ e $P''\ped{xy}$), sia (u,v) il primo arco di $P\ped{xy}$ che non appartiene a $P''\ped{xy}$ \\
$T \setminus (uv)$ non è connesso, u e v appartengono a differenti componenti connesse. \emph{ma} in $T \setminus (uv)$ esiste un persorso u - v. \\
$b \Rightarrow a$ ) T tale che \(\forall x,y \in V, x \neq y\) esiste un unico cammino.
Supponiamo che C sia un circuito di T. Siano x e y $\in$ C, x $\neq$ y. \\
Esistono due cammini distinti che collegano x e y.
\\

\subsection{Albero di peso minimo. Algoritmo di Kruskal.}
\textbf{Problema:} Dato G(V,E) connesso, con pesi \( w(e), e \in E\) \\
\textbf{input:} G(V,E) connesso, pesi \\
\textbf{output:} Albero di peso minimo.\\
\begin{itemize}
\item \emph{Ordina} gli archi ${1, 2, .., m}$ in ordine nondecrescente di peso: \( w(1)\leq w(2)\leq .. \leq w(m)\)
\item \emph{Inizializzazione:} Poni $T=\emptyset$ ed $i=1$.
\item \emph{Iterazione i:} Se l'arco i ha le estremità in componenti connesse distinte di T, allora poni
\(T \leftarrow T \cup \{i\} \). Altrimenti \(T \leftarrow T\). Poni \(i \leftarrow i+1\) e ripeti l'iterazione.
\item \emph{Stop:} Quando $i=m$.
\end{itemize}

\subsection{Esercizi}
\subsubsection{Es.1}
Dimostrare che ogni grafo non orientato semplice G(V,E), senza cicli e con \(|E|=|V|-1\) è un albero. \\
\textbf{Svolgimento:} \\
Devo dimostrare che G(V,E) è connesso. Sia $c$ il numero di componenti connesse di G. \\
\(G_1(V_1,E_1)\) connesso e senza circuiti $\rightarrow$ albero; \( |E_1| = |V_1| -1 \) \\
\(G_c(V_c,E_c) \)connesso e senza circuiti $\rightarrow$ albero; \( |E_c| = |V_c| -1 \) \\
Sommando: \(|E|=|E_1|+ |E_2| + .. + |E_c| = |V_1|-1 + |V_2|-1 + .. + |V_c|-1 = |V| - c\) \\
Ma: \(|E| = |V| -1\) per ipotesi $\Rightarrow c=1 \Rightarrow$ G connesso $\Rightarrow$ è un albero.
%.....

\subsubsection{Es.2}
Dimostrare che ogni grafo non orientato semplice e connesso con $n$ vertici e $n-1$ archi,
è un albero.
%....

\subsubsection{Es.3}
Dimostrare che data una foresta F(V,E), vale: \(\mid E \mid = \mid V \mid - \gamma \) , con $\gamma =$ numero di componenti connesse di F(V,E).
%...

\subsubsection{Es.4}
\textbf{Problema:} Dato un grafo connesso con lunghezze positive sugli archi, e vertici r, v,
\emph{trovare} un cammino di lunghezza minima fra r e v. (lunghezza di un cammino $P =$ somma delle
lunghezze degli archi in P, lunghezza minima fra r e $v=$ distanza fra r e v). \\
Uso l'algoritmo di \emph{Dijkstra}. \\
\textbf{Input}: G(V,E) connesso, lunghezze \(l(e) > 0, e \in E\), non di partenza r \\
\textbf{output:} distanze (e cammini minimi) fra r e gli altri nodi in V.\\
\emph{Inizzializzazione:} Poni d(r) $=$ 0, d(v)$= \inf$, \(v \in V \setminus \{r \} i= 0, S_0 = \{r \}\). \\
\emph{Iterazione i:} \(\forall v \in V \setminus S_i\) poni \(d(v)= \min \{ d(v), d(u)+l(uv) \}\). \\
Sia $v \in V \setminus S_i$ il vertice per cui d(v) è minimo. \\
Poni \(i=i +1, S_i=S\ped{i-1} \cup \{ v\}\). \\
\emph{Stop:} Quando $i= |V-1|$.
%...


\newpage
