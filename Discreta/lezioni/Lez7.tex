\section{Settima lezione: I circuiti hamiltoniani}
\subsection{Circuiti hamiltoniani}
Dato G(V,E) grafo, un (circuito) ciclo Hamiltoniano di G è un ciclo che visita \textbf{ogni vertice} esattamente
\textbf{una volta}. \\
Il grafo G viene detto Hamiltoniano se ha un ciclo Hamiltoniano. \\
Ricordiamo che abbiamo visto un algoritmo per testare la planarità di un grafo, conoscendo un ciclo
Hamiltoniano. \\
Stabilire se un grafo è Hamiltoniano è un problema difficile (NP-completo): appartiene ad una classe di
problemi per cui non esistono algoritmi semplici e veloci che risolvano qualsiasi istanza del problema. \\
Esistono condizioni \emph{necessarie} (che ogni grafo deve soddisfare per essere Hamiltoniano) e \emph{sufficienti} (che assicurano che un grafo che le soddisfi sia Hamiltoniano, ma non viceversa.) ma non esistono condizioni semplici che siano \emph{necessarie} e \emph{sufficienti} (contemporaneamente).

\subsubsection{Condizioni necessarie}
Osserviamo che: ogni condizione che è verificata da un ciclo Hamiltoniano H deve essere verificata da un grafo
che contiene H. \\
Se G(V,E) è Hamiltoniano, allora:
\begin{itemize}
    \item \(d(v) \geq 2  \forall v \in V\)
    \item \(K(G) \geq 2 \)
    \item Più \emph{importante}: per ogni sottoinsieme S di V, abbiamo \( |S| \geq \) delle componenti connesse di \(G \setminus S\).
\end{itemize}

\subsubsection{Condizioni sufficienti per l'esistenza di un ciclo hamiltoniano}
\paragraph{Teorema di Dirac} Sia G(V,E) un grafo semplice, con n vertici, \(n>2\), se \(d(v) \geq n/2\) per ogni vertice \(v \in V\), allora il grafo G è Hamiltoniano. \\
La condizione di Dirac è \emph{sufficiente} ma \emph{\textbf{non} necessaria}.

\subsection{Regole per costuire un ciclo Hamiltoniano}
Dato G (V,E), le seguenti regole possono aiutare nella ricerca di un ciclo hamiltoniano (CH), se
esiste. \\
Se \( K(G) \leq 1\), allora G non contiene nessun CH.\\
\begin{enumerate}
    \item Se un vertice ha grado 2, allora i due archi incidenti in esso devono appartenere a qualsiasi CH.
    \item Un ciclo Hamiltoniano non può contenere sottocicli propri.
    \item Se nella costruzione di un CH abbiamo individuato due archi incidenti nel vertice v e
    appartenenti al CH, allora tutti gli altri archi incidenti in v non possono appartenere a CH e
    possone essere rimossi per la ricerca del CH.
\end{enumerate}
Ricordiamo che: dato G(V,E), un ciclo è \emph{hamiltoniano} se contiene
tutti i vertici di G. \\ 
G è \emph{hamiltoniano} se contiene un ciclo hamiltoniano.\\ 
%Grafi hamiltoniani: Kn, n>=3 Kn,n n>=2 ipercubi
%Grafi nonhamiltoniani: Kn,m n diverso da m, Petersen. Scacchiera.

\noindent
\subsection{Percorsi euleriani}
Ricordiamo che: dato G(V,E), un \emph{percorso} è una sequenza v1, e1, v2, e2, v3, e3 ... vn, en, vn+1 (con possibile
ripetizione di vertici o archi) dove \(ei=vi,vi+1\). \\ 
Il percorso è \textbf{chiuso} se \(v1=vn+1\).\\ 
Il percorso è \emph{euleriano} se è chiuso e contiene \emph{esattamente una volta} tutti gli archi di G(V,E) (ma i vertici
possono essere attraversati più di una volta). \\ 
G(V,E) è \emph{euleriano} se contiene un percorso euleriano. 
\paragraph{Teorema di Eulero} G(V,E) è euleriano se e solo se G è connesso ed ogni vertice di G ha grado pari. 
\subsubsection{Algoritmo per trovare un percorso euleriano}
Parti da un nodo
arbitrario e inizia a percorrere il grafo (rimuovendo gli archi già percorsi). Percorri un arco che non
appartiene ad un ciclo solo se è l'unico che puoi percorrere. \\





\newpage
