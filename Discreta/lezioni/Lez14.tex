\section{Quattordicesima lezione: Relazioni dividi e conquista}
\( \left\{ 
    \begin{array}{rcl}
    a_n = c* a\ped{n/2} + f(n) \\
     \mbox{condizione iniziale }
    \end{array}
     \right. \)
\paragraph{Caso 1:} \(c =1, f(n)=d\) , \( a_n = a\ped{n/2}+ d\) \\
Soluzione generale: \(a_n = d [\log_2 n ]+ A\) con $n>0$ \\ 
Verifica: \\
\paragraph{Caso 2:}   \(c=2, f(n)  =d\) \\
Soluzione generale: \\
Verifica: \\

\subsection{Esempi ed esercizi}
\paragraph{Esempio: torneo di tennis} $a_n =$ numero di turni di un torneo di tennis con n giocatori, dove $n$ è una potenza di 2.





\newpage