\section{Terza lezione: Connettività}

Ricordiamo: dato G(V,E) u,v sono connessi se esiste un cammino con estremità u, v. \\
La CONNESSIONE ha le seguenti proprietà:
\begin{itemize}
\item u è connesso a se stesso \( \rightarrow \) \emph{riflessività}
\item u,v connessi, v,u connessi \( \rightarrow \) \emph{simmetria} 
\item u,t connessi, t,v, connessi, allora u,v connessi \( \rightarrow \) \emph{transitività}
\end{itemize}
Queste proprietà implicano che esiste una \emph{partizione} di V in parti \(V1,...Vk \) dove: 
u,v sono connessi se appartengono alla stessa parte. \\
Queste parti sono le COMPONENTI CONNESSE di G. \\
G è \emph{connesso} se c'è una sola parte! \((k=1)\). \\

\noindent
Dato \( S \subseteq V \) definiamo il \emph{taglio} associato ad S come l'insieme di archi. \\
\( \delta(S) = \{ \{uv\} \in E , \mid S \cap \{u, v\} \mid = 1\} \) \\
e diciamo che $\delta(S)$ \emph{separa} $u,v$ se \( \mid S \cap u,v \mid =1 \) \smallskip
\paragraph{Teorema} 
Dato G(V,E), vertici $u,v$ appartengono alla stessa componete connessa di G se e solo se non esiste un taglio $\delta(S) = \emptyset $
Dimostriamo che: dato $\delta(S)$ che separa $u,v$ e $P$ cammino fra $u,v$ allora $P$ e $\delta(S)$ hanno almeno un arco in comune, cioè \(\mid P \cap \delta(S) \mid \geq 1\)\\
Se $\delta(S) \neq \emptyset$ per ogni taglio $\delta(S)$ che separa $u,v$ allora esiste un cammino fra $u,v$. \\

\subsection{Calcolo delle componenti connesse}

\textbf{Input: }G(V,E), v in V \\
\textbf{Output:} Componente connessa C che contiene v \smallskip
\begin{itemize}
\item Poni $v \in C$ e dichiara $v$ non esaminato
\item \emph{esame vertice:} scegli $x \in C$ non esaminato. Aggiungi a $C$ ongi vertice adiacente a $x$ che non è in $C$.
\item \emph{stop} quando ogni vertice in $C$ è esaminato.
\item $C$ è la componente connessa che contiene $v$.
\end{itemize}
Sia $C$ componente connessa che contiene $v$ calcolata dall'algoritmo. Si noti che \(\delta(S) = \emptyset \) \\
Se $C=V$ il grafo è connesso. Se $C$ è strettamente contenuto in $V$, sia $u$ in $V$ o $C$ allora \(\delta(C)\) separa $u$ e $v$.\\

\subsection{Connettività}
Ricordiamo: un grafo non orientato G(V,E) si dice \emph{connesso} se, per ogni coppia di vertici, esiste
un cammino che li collega, altrimenti si dice \emph{disconnesso}. \\
Le componenti connesse di un grafo sono i suoi sottografi connessi massimali. 
\paragraph{Connettività sugli archi}
Arco connettività fra \( u,v = K^E_{uv} (G)\) 

\smallskip
\noindent
Cardinalità minima di un taglio che separa \( u,v = min \{ \mid \delta(S) \mid : S \) separa \( u,v \}\) \\
Ricordiamo: dato G(V,E) u,v in componenti connesse distinte se e solo se \(K^E_{uv} (G) =0\) 

\smallskip
\noindent
Conseguenza: \(K^E_{uv} (G)\) è il minimo numero di archi da togliere a G affinchè u,v diventino disconnessi. 

\smallskip
\paragraph{Teorema} Per ogni \(u,v K^E_{uv} (G)\) è il massimo numero di cammini con estremità $u,v$ che non hanno archi in comune (disgiunti sugli archi). 

\medskip
\noindent
Ricordiamo che: G(V,E) connesso se e solo se non esiste \( \delta(S)\) che separa due vertici tali che \( \delta(S) = \emptyset \) 

\medskip
\noindent
Definiamo: arco connettività di G $K^E(G) =$ minimo numero di archi da togliere affinchè G sia disconnesso $=$ cardinalità minima di un taglio che separa almeno 2 vertici. \\
Allora: \( K^E(G) = \min K^E_{uv} (G)\)

\subsubsection{Connettività (sui vertici)}

Dato G(V,E) grafo connesso, semplice e non completo: \\
\emph{Connettività (sui vertici) K(G)} $=$ minimo numero di vertici la cui rimozione trasforma G in un grafo
sconnesso ($=$ non connesso).

\medskip
Vertex cutset = insieme di vertici $U \subseteq V $con le due seguenti proprietà:
\begin{itemize}
\item la rimozione di tutti i vertici di U trasforma G in un grafo sconnesso;
\item la rimozione di alcuni dei vertici di U, ma non tutti, lascia il grafo connesso.
\end{itemize}

\noindent
Quindi, se G non è completo: \(K(G) = \) cardinalità del vertex cutset di G di cardinalità minima.\\
Un grafo completo con n vertici non ha vertex cutsets.
Per convenzione, la connettività sui vertici di un grafo completo con n vertici è posta uguale a n-1.\\

\noindent
Dato G(V,E) e i vertici $u,v$ \emph{non adiacenti}, \(K\ped{uv}(G)\) cardinalità minima di un separatore S tale che $u,v$ sono in componenti connesse distinte di G/S. \\
Quindi se G(V,E) è un grafo semplice non completo allora: \(K(G)= \min K\ped{uv}(G) u,v\) non adiacenti. \\
Dato G(V,E) e vertici $u,v$ non adiacenti, P un cammino con estremità $u,v$, S un separatore con $u,v$ in componenti connesse distinte di G/S, allora \emph{S contiene almeno un vertice intermedio di P}\\

\subsubsection{Disequazioni}
\paragraph{Teorema} Sia G(V,E) grafo con almeno due vertici. Allora:
\[ K(G) \leq K^E(G) \leq \delta(G) \]
dove il simbolo d(G) indica il grado del vertice di grado minimo di G: \(\delta(G) = \min {d(v) v \in V} \)  \\
In particolare, se G è il grafo completo con n vertici, $n \geq 2$, allora \(K(G) = K^E(G) = \delta(G) =n-1\) \\
Se $G = K_n$ allora \(K(G) = n-1 \leq  K(G) \leq \delta(G) = n-1\), allora \(\lambda(G) =n-1 \) \\

\subsection{Affidabilità di una rete}

Rete di comunicazione tra soggetti: è necessario fornire percorsi alternativi di comunicazione, per fronteggiare:
\begin{itemize}
\item inattività di un soggetto;
\item guasto di una linea;
\item carico eccessivo di una linea, che superi la sua capacità.
\end{itemize}
Dato G grafo:
\begin{itemize}
\item per ogni coppia di nodi esistono almeno Ke(G) cammini alternativi che li collegano (tali che non ci
siano due cammini che usano qualche arco in comune);
\item per ogni coppia di nodi esistono almeno K(G) cammini alternativi che li collegano (tali che non ci
siano due cammini che usano qualche vertice in comune).
\end{itemize}
In generale vale la seguente formula: \\
\[K(G) \leq Ke(G) \leq d(G) \leq 2 \frac{ \mid E \mid}{\mid V \mid} \] grado medio in $G =$ media \[ \{ gr(v) : v \in V \} = \frac{\sum(v \in V) gr(v)}{ \mid V \mid} = \frac{2 \mid E \mid}{\mid V \mid} \] \\
Si dice che un grafo ha connettività ottima se vale \[ K(G) = Ke(g) = d(G) = 2 \frac{\mid E \mid}{\mid V \mid} \] \\
Proprietà:
\begin{enumerate}
\item ha massima connettività sugli archi e sui vertici tra tutti i grafi con $\mid V \mid$ vertici e $\mid E \mid$ archi
\item è $\delta$ -regolare
\end{enumerate}

\(K^E_{uv} (G) =\) numero massimo di cammini con estremità $u,v$ che non hanno archi in comune. \\
\(K_{uv} (G) =\) numero massimo di cammini con estremità $u,v$ che non hanno nodi intermedi in comune. \\

\paragraph{Dimostrazione} Dimostrare che non può esistere nessun taglio del grafo di Peterson con 8 archi. \\
G(V,E) grafo di peterson, \(\mid V \mid = 10 , \mid E \mid = 15\). \\
Suppongo che esista un taglio $\delta(S)$ con 8 archi, quindi:
\begin{itemize}
\item G(V,E $\setminus$ E') è sconnesso
\item G(V,E $\setminus$ E'') è connesso \( \forall E'' \subsetneq E'\)
\end{itemize}
Sia $E'' \subsetneq E'$, con $\mid E'' \mid = 7$ \\
G(V,E $\setminus$ E'') quindi dovrebbe essere connesso, ma questo è impossibile avendo 10 vertici e $15-7=8$ archi (il grafo minimalmente connesso con 10 vertici è un albero, e ha 9 archi).

\newpage
