\section{Undicesima lezione: Disposizioni con ripetizione}

Le disposizioni con ripetizioni sono tutti i possibili ordinamenti di oggetti, alcuni dei quali indistinguibili. 

\noindent
\textbf{Esempio:} Quante sono le disposizioni delle lettere della parola BANANA? \\
Le lettere da assegnare ai 6 posti disponibili sono una B, due N e tre A. Avrei delle combinazioni quali
\smallskip
\( \binom{6}{1}, \binom{5}{2}, \binom{3}{3} = \frac{6!}{1! 5!} \frac{5!}{2! 3!} \frac{3!}{3! 0!} = \frac{6!}{1! 2! 3!}\)

\paragraph{Teorema} 
Abbiamo n oggetti, di cui
\begin{itemize}
    \item $r_1$ del tipo 1 (identici)
    \item $r_2$ del tipo 2 (identici)
    \item $\dots$
    \item $r_m$ del tipo m (identici)
\end{itemize}
con $r_1 + r_2 + ... + r_m = n$. \\
Il numero delle disposizioni di questi n oggetti è:\\
\smallskip
\( \binom{n}{r_1} \binom{n-r_1}{r_2} \binom{n-r_1-r_2}{r_3} \dots  \binom{r_m}{r_m} = \frac{n!}{r_1 ! r_2 ! \dots r_m !} \) \\
Numero di disposizioni con ripetizione 
%dimostrazione


\subsection{Selezioni con ripetizione}
\textit{Esempio:} In quanti modi possiamo selezionare 6 caffè tra le varietà: liscio, macchiato, lungo? \\


\paragraph{Teorema}
Il numero di selezioni con ripetizione di r oggetti scelti tra n varietà è: \\
\smallskip
\( \binom{r + n -1}{r}= \binom{r+n-1}{n-1}= \) numero di selezioni con ripetizione \\
= numero di soluzioni intere della equazione
\[ \left\{ 
    \begin{array}{rcl} 
    x_1 + x_2 + \dots + x_n = r \\
    x_1, x_2, \dots, x_n \geq 0
\end{array} 
\right\} \]

\textit{Esempio} \\ %2 
$x_1 = $numero di caffè lisci \\
$x_2 = $numero di caffè macchiati \\
$x_3 = $numero di caffè lunghi \\
Le selezioni di 6 caffè tra 3 varietà sono tante quante le soluzioni intere della equazione

\paragraph{Esempio} %5
Quante sequenze di lunghezza 10 si possono formare scegliendo gli elementi tra
quattro "A",
quattro "B",
quattro "C",
quattro "D",
se ogni lettera deve apparire almeno due volte?

\paragraph{Esempio} %6
In una pasticceria, ci sono 5 varietà di dolci. In quanti modi possiamo scegliere 12
dolci tra queste 5 varietà, se vogliamo prendere almeno un dolce per varietà?

\subsection{Distribuzioni di oggetti distinti} 
r oggetti distinti, n scatole distinte: Quanti sono i modi di porre questi r oggetti nelle scatole, se
una scatola puo' contenere piu' di un oggetto?
 distribuzioni di r oggetti distinti in n scatole distinte
 stringhe di lunghezza r i cui elementi sono scelti in
{1, ... ,n} con ripetizione :

 distribuzioni di r oggetti distinti in n scatole distinte,
sapendo che nella scatola i devono entrare $r_i$ oggetti
($i=1, ... ,n$) con $r_1 + r_2 + ...+ r_n = r$
 disposizioni di r oggetti, di cui $r_1 $ del tipo 1, ........ ,

\subsection{Distribuzioni di oggetti identici}
r oggetti identici, n scatole distinte
 distribuzioni di r oggetti identici in n scatole diverse
 selezioni con ripetizione di r oggetti scelti tra n varietà

\paragraph{Esempio} %5
\begin{enumerate}
    \item Il numero di modi di distribuire r palline identiche in n scatole distinte.
Il numero di modi di distribuire r palline identiche in n scatole distinte,
con almeno una pallina per scatola ($r>=n$)
    \item Il numero di modi di distribuire r palline identiche in n scatole distinte,
con almeno $r_1$ palline nella prima scatola, almeno $r_2$ nella seconda, ... ,
almeno $r_n$ nella n-esima
\end{enumerate}

\paragraph{Esempio} %7
Vado al mercato con 5 euro da spendere. Un cestino di fragole costa 1.5 euro,
una banana o una mela o una arancia costano 50 cent. In quanti modi posso
spendere i miei soldi, supponendo di volerli spendere tutti?


\subsection{Esercizi}

\paragraph{Es.1} Quante diverse sequenze di lunghezza cinque si possono formare usando le lettere
A, B, C, D con ripetizione, con la condizione che le sequenze non contengano la parola BAD
(ad esempio, la sequenza ABADD è esclusa)?

\paragraph{Es.2} In quanti modi una persona può invitare un sottoinsieme di almeno 3 dei suoi 10 amici a cena?

\paragraph{Es.3} Quante sequenze di cinque caratteri (scelti tra le 26 lettere dell'alfabeto, anche ripetute)
contengono esattamente una A ed esattamente due B?

\paragraph{Es.4}
\begin{enumerate}
    \item Quante sequenze di dieci lettere si possono formare usando 5 differenti vocali
e 5 differenti consonanti (scelte tra le 21 consonanti)?
\item Tra queste, quante sono quelle che non hanno coppie consecutive di
consonanti e non hanno coppie consecutive di vocali?
\end{enumerate}

\paragraph{Es.5}
\begin{enumerate}
    \item Quanti diversi triangoli si possono formare congiungendo terne di vertici di un
ottagono (regolare)?
\item Quanti diversi triangoli si possono formare se si vieta di utilizzare terne che
contengano due vertici adiacenti nell'ottagono?
\item Quanti diversi triangoli si possono formare che abbiano due lati coincidenti con lati
dell'ottagono?
\item Quanti diversi triangoli si possono formare che abbiano un solo lato coincidente
con lati dell'ottagono?
\end{enumerate}

\paragraph{Es.6}
\begin{enumerate}
    \item Quanti numeri di sette cifre si possono formare con le cifre 3, 5 e 7?
\item Di questi numeri, quanti hanno tre cifre uguali a 3, due uguali a 5 e due uguali a 7?
\end{enumerate}




\newpage