
    \section{Prima lezione: Introduzione}

    \subsection{Teoria dei grafi}
        Grafo = G(V,E) \\
        \textbf{Planarità di un circuito stampato:} 
Nei circuiti stampati, le componenti elettroniche sono collegate da piste conduttrici stampate su una tavoletta
isolante; queste piste non si possono incrociare.\\

\textsc{Problema del postino cinese} \\
Un postino deve consegnare la posta nelle seguenti strade. Può partire da A e tornare in A senza percorrere due volte
la medesima strada ? \\
Equivalente: Disegnare la seguente figura senza sollevare la penna dal foglio (partendo e tornando in un punto
arbitrario) \\

\textsc{Spelling checker} \\
Cercare una parola nel dizionario. Zucca: si ricorre all'uso degli alberi binari \\

\textsc{Torneo ad eliminazione} \\
16 Giocatori: Torneo ad eliminazione diretta. Anche in questo caso si usano gli alberi binari.

\subsection{Metodi di conteggio} 
Contare un numero (finito) di oggetti non è sempre facile ma è un problema che si pone spesso in
informatica: per esempio contare le operazioni fatte da un algoritmo per misurarne l'efficienza. \\
\begin{center}
\textsc{Metodi di conteggio $=$ Enumerazione $=$ Calcolo combinatorio}
\end{center} 

\textit{Esempi:} 
\begin{itemize}
        \item Numero di parole con un fissato alfabeto
\item Numero di sequenze binarie, numero di sottoinsiemi di n elementi
\item Numero di soluzioni intere di una data equazione lineare
\item Numero di targhe automobilistiche
\item Probabilità discreta: Quale è la probabilità che 4 carte diano un poker? (d'assi?)
\end{itemize}

\subsection{Relazioni di ricorrenza}
Spesso è difficile calcolare direttamente una quantità (come il numero \( K\ped{n}\)
di sottoinsiemi di n
elementi con una data proprietà). E' più semplice trovare una relazione di ricorrenza che
determina \( K\ped{n} \) in funzione di \( K\ped{n-1} \). \\
\begin{center}
\textsc{Relazioni di ricorrenza $=$ Equazioni alle differenze finite}
\end{center}

Problemi ricorsivi:
\begin{enumerate}
\item Calcolo del determinante di una matrice quadrata
\item Numeri di Fibonacci
\item Calcolo di n!
\item Torre di Hanoi
\end{enumerate} 

\texttt{Esempio 1.1:} \par
    \(a\ped{n} = a\ped{n-1} * n \rightarrow \) \textsc{relazione di ricorrenza} \par
    \(a\ped{0} = 1 \rightarrow \) \textsc{condizione iniziale} \\

    \( a\ped{n} = n! \rightarrow \) \textsc{soluzione} \\

\texttt{Esempio 1.2:} \\
\textbf{Evoluzione di un capitale investito in banca:}  All'anno 0 investo 1000EUR. Il mio investimento dà un interesse del 4 \% annuo. Dopo n anni quanto e' il mio capitale? \\

\(a\ped{0} = 1000 \) \par
\(a\ped{1} = 1000 + \frac{4}{100} * 1000 = 1.04 * a\ped{0} \) \par
\(a\ped{n} = (1.04) * a\ped{n-1} \rightarrow \) \textsc{relazione di ricorrenza} \par

\(a\ped{2} = 1.04 * a\ped{1} = (1.04)^2 * a\ped{0} \) \par

\(a\ped{n} = (1.04)^n * a\ped{0} \rightarrow \) \textsc{soluzione} 
\\

\texttt{Esempio 2:} \\
\textbf{Calcolare il massimo di una stringa ($x_1$, .. , $x_n$ ) di n numeri} \\
$a_n$ $=$ numero di confronti necessari per calcolare il massimo di una stringa di lunghezza n \\

\( a\ped{n} = a\ped{n/2} + a\ped{n/2} +1 \rightarrow \) \textsc{relazione di ricorrenza} \par

\( a\ped{2} = 1 \rightarrow \) \textsc{soluzione} \par 

\newpage
