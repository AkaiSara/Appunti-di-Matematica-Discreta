\section{Quarta lezione: Grafi bipartiti e isomorfi}
\subsection{Grafi bipartiti}

Ricordiamo che: un grafo G(V,E) è bipartito se il suo insieme di vertici può essere partizionato in due
sottoinsiemi, $V_1$ e $V_2$, in modo che ogni arco ha un estremità in $V1$ e l'altra in $V2$. Scriveremo G($V_1$, $V_2$; E). \\
Massimo numero di archi in un grafo bipartito semplice: \( \mid V_1 \mid * \mid V_2 \mid \) 

\subsubsection{Condizione necessaria E sufficiente} 
\textbf{Osservazione 1:} Se G(V,E) è bipartito e G'(V',E') sottografo di G, allora G'(V',E') bipartito. \\
Ovvero:
G'(V',E') sottografo di G(V,E), G' è nonbipartito, allora anche G è nonbipartito. \\
\textbf{Osservazione 2:} Se G(V,E) è bipartito con bipartizione V1 e V2, sia v un vertice. Posso assumere senza perdita di generalità v in V1. Allora tutti I vertici adiacenti a v DEVONO stare in V2. \\
Cioe' se v in V1, allora l' arco vu "forza" u in V2.\\
\textbf{Osservazione 3:} Se G(V,E) contiene un ciclo dispari come sottografo, allora G(V,E) \emph{non} è bipartito.\\
\paragraph{Teorema:} 
Un grafo G(V,E) è bipartito se e solo se G non contiene un ciclo di lunghezza dispari. \\
Per quanto detto sopra, se G contiene un ciclo dispari, G \emph{non} è bipartito. \\
Dimostriamo che se G \emph{non} contiene un ciclo dispari, allora G è bipartito. Scegliamo v in V. Sia:
\begin{itemize}
\item V1 l'insieme dei nodi di G raggiungibili da v con un cammino di lunghezza pari.
\item V2 l'insieme dei nodi di G raggiungibili da v con un cammino di lunghezza dispari.
\end{itemize}
\emph{N.B.} Se G è bipartito allora \(V1 \cap V2 \)  $= \emptyset $ e nessun arco di G ha entrambe le estremità in V1 o in V2.

\subsubsection{Algoritmo per bipartizione}
\textbf{Input:} Un grafo G(V,E) \\
\textbf{Output:} Se G è bipartito, una bipartizione R, B di G. \\
Per ogni componente connessa di G(V,E):
\begin{itemize}
\item Poni \(v \in B , R = \emptyset \)
\item \emph{esame vertice} $v \in B (R)$. Aggiungi a $R (B)$ ogni vertice adiacente a $v$ non in $R \cup B$
\item \emph{stop} quando ogni vertice di $R \cup B$ è esaminato
\item G(V,E) è bipartito se e solo se ogni arco $e$ ha un estremità ub $R$ ed una in $B$
\end{itemize}



\subsection{Grafi isomorfi}

Due grafi G(V,E) e G'(V',E') sono isomorfi se esiste una corrispondenza biunivoca (isomorfismo) tra i
vertici di V e quelli di V' tale che: \\
due vertici di V sono adiacenti in G se e solo se i corrispondenti vertici di V' sono adiacenti in G'. \\
\paragraph{Determinare se due grafi sono isomorfi:}
Per farlo bisogna cercare l'isomorfismo. \\
Se due grafi sono isomorfi:
\begin{itemize}
\item hanno lo stesso numero di \emph{vertici}
\item hanno lo stesso numero di \emph{archi}
\item hanno lo stesso numero di vertici con lo \emph{stesso grado}
\end{itemize}

\subsubsection{Condizioni necessarie}
Due grafi sono isomorfi se:
\begin{itemize}
\item hanno lo stesso numero di \emph{vertici}
\item hanno lo stesso numero di \emph{archi}
\item hanno lo stesso numero di vertici con lo \emph{stesso grado}
\item i complementari devono essere isomorfi
\item hanno gli stessi sottografi indotti.
\end{itemize}
Se le prime 3 condizioni sono verificate, costruisco il possibile isomorfismo accoppiando vertici dello
stesso grado, controllando che la condizione 5 via verificata. \\

%ESERCIZIO: Dimostrare che se G(V,E) soddisfa 3), allora G soddisfa anche 1) e 2)

%1) Sia G(V,E) un grafo 2-regolare. Dimostrare che ogni componente connessa di $G$ e' un ciclo.
%Sia G(V,E) un grafo con
%Dimostrare che ogni componente connessa di G e' un
%cammino o un ciclo. (un nodo isolato e' considerato un cammino di lunghezza 0).
%2) Siano G1 e G2 due grafi 2-regolari. Date un algoritmo per testare se sono isomorfi.
%3) In grafo seguente e' bipartito? (applicate l'algoritmo)






\newpage
