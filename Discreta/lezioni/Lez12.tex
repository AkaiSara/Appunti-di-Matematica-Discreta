\section{Dodicesima lezione: Esercizi}

\subsection{Esercizi}

\paragraph{Es. 7} Nove persone entrano in un bar per acquistare un panino per ciascuno.
Tre ordinano sempre il panino con il tonno,
due sempre quello con le uova,
e due sempre quello con il prosciutto.
Le rimanenti due persone ordinano ciascuna uno qualsiasi dei tre tipi di panino ordinati dagli altri.
Quante diverse selezioni non ordinate di panini sono possibili?

\paragraph{Es. 8}Ordinando le cifre 1, 2, 2, 4, 4, 6, 6 a caso, scriviamo un numero di sette cifre.
\begin{enumerate}
    \item Quanti diversi numeri maggiori di 3000000 (tre milioni) possiamo scrivere?
     \item Quanti numeri dispari maggiori di tre milioni possiamo scrivere?
\end{enumerate}

\paragraph{Es. 9}In quanti modi differenti si possono disporre le lettere della parola REPETITION
in cui la prima E compare prima della prima T?

\paragraph{Es. 10} In quanti modi si possono distribuire 36 caramelle identiche a quattro bambini:
\begin{enumerate}
    \item senza restrizioni
    \item in modo che ogni bambino riceva 9 caramelle
    \item in modo che ogni bambino riceva almeno una caramella
\end{enumerate}

\paragraph{Es. 11} In quanti modi si possono distribuire
7 identiche mele, 6 identiche arance e 7 identiche pere
fra tre persone diverse:
\begin{enumerate}
    \item senza restrizioni
    \item in modo che ogni persona riceva almeno una pera
\end{enumerate}

\paragraph{Es. 12} Dire quante sono le soluzioni intere della equazione
con le condizioni:

\paragraph{Es. 13} Dire in quanti modi si possono distribuire k palline in n scatole distinte (k<n) con al massimo
una pallina per scatola se:
\begin{enumerate}
    \item le palline sono distinte
    \item le palline sono identiche
\end{enumerate}

\paragraph{Es. 14}
In quanti modi si possono suddividere tra due scatole
4 palline rosse identiche, 5 palline blu identiche e 7 palline nere identiche, se:
\begin{enumerate}
    \item non ci sono restrizioni
    \item si vuole che nessuna scatola resti vuota
\end{enumerate}


\newpage