\section{Esercizi svolti}


\paragraph{Es.1} 
Dimostrare che, se un grafo connesso e bipartito G(V1, V2; E) ha un ciclo hamiltoniano, allora
le cardinalità di V1 e di V2 devono essere uguali.
Inoltre, se un grafo bipartito ha un numero dispari di vertici, allora non può avere cicli
hamiltoniani. \\

\noindent
\textsc{Risposta:} Sia \(c = (x_1, x_2, \dots , x_n, x_1) \) un circuiti Hamiltoniano in G. \\
\(x_1 \in V_1 \Rightarrow x_2 \in V_2\) \\
\(x_3 \in V_1 \Rightarrow x_4 \in V_2\) \\
\(x_5 \in V_1 \Rightarrow x_6 \in V_2\) \\
$\dots$ \\
\(\exists (x_1,x_n) \in E \Rightarrow x_n \in V_2 \Rightarrow n\) pari \(\Rightarrow |V_1| = |V_2| \) \\
Se \(\exists c\) circuito Hamiltoniano $\Rightarrow n$ pari. 

\paragraph{Es.2}
\begin{enumerate}
    \item Quanti diversi cicli Hamiltoniani ci sono nel grafo \(K_n, n \geq 3\), il grafo completo con n vertici?
    \item Dimostrare che gli archi del grafo \(K_n\), con n primo, si possono partizionare in \(\frac{1}{2}(n-1)\) hamiltoniani disgiunti sugli archi.
    \item Se 11 persone cenano insieme ogni sera disponendosi attorno ad una tavola rotonda, e ogni
    sera ogni persona si vuole sedere tra due persone che non gli sono mai state vicino, quante cene
    riescono a fare?
\end{enumerate}
\textsc{Risposta:}

\paragraph{Es.3} Il seguente grafo è noto come grafo di Petersen.\\
E' un grafo con delle proprietà particolari e non è facile da studiare, pur avendo solo 10 vertici. \\
E' bipartito? Hamiltoniano? Planare? \\

\noindent
\textsc{Risposta:} \\
\textbf{Oss:} non è bipartito e non è hamiltoniano. \\
Non si può applicare il metodo del cerchio e delle corde perchè non c'è un ciclo hamiltoniano. \\
La disequazione \(e \leq 3v -6\) è soddisfatta. \\
Quindi possiamo solo cercare un $K \ped{3,3}$ (non può essere un $K_5$ perchè servirebbero 5 vertici non di grado 4). 

\paragraph{Es.4} Rispondere alle seguenti domande, motivando le risposte.
\begin{enumerate}
    \item Qual è il massimo numero di archi in un grafo non orientato semplice con 15 vertici?
    \item Qual è il massimo numero di archi in un grafo orientato semplice con 15 vertici?
    \item Qual è il minimo numero di vertici in un grafo non orientato semplice con 55 archi? E se ha 60 archi?
\end{enumerate}
\textsc{Risposta:}
\begin{enumerate}
    \item \( G(V,E)\)  \(|E| \leq \frac{|V|(|V|-1)}{2} = \frac{15*14}{2} =105 \)
    \item \(|E| \leq |V|(|V|-1) = 15*14=210\)
    \item \( G(V,E) |E|=55\) 
\end{enumerate}
Esiste un grafo completo $K_n$ con 55 archi? Se si, quanto deve valere n?\\

\noindent
\textsc{Risposta:} \\
\(|E(K_n)| = \frac{n(n-1)}{2}= 55\) \\
\(n(n-1)= 110\) \\
\(n^2 - n- 110=0\) \\
\(n= \frac{1 \pm \sqrt{1+440}}{2}= \frac{1 \pm \sqrt{441}}{2} = \frac{1 \pm 21}{2}=11 \) \\
Quindi $K\ped{11}$ ha 55 archi ed è il grafo con il minor numero di vertici tra quelli che hanno 55 archi. \\
Se \(|E|=60 \Rightarrow \) allora è necessario almeno un vertice in più \(\Rightarrow |V| \geq 12\). \\
Esistono grafi con 12 vertici e 60 archi? \\
\(|E(K\ped{12})| = \frac{12*11}{2}= 66 \Rightarrow\) quindi \(|V|= 12\)

\paragraph{Es.5} Qual è il massimo numero possibile di vertici in un grafo con 19 archi e tutti i vertici di grado maggiore o uguale a 3?\\

\noindent
\textsc{Risposta:}

\paragraph{Es.6} Se un grafo non orientato ha n vertici, tutti di grado dispari
tranne uno, quanti vertici di grado dispari ci sono nel suo complementare?\\

\noindent
\textsc{Risposta:}

\paragraph{Es.7} Determinare se i seguenti grafi sono bipartiti:\\

\noindent
\textsc{Risposta:}

\paragraph{Es.8} Rispondere alle seguenti domande, motivando le risposte.
\begin{itemize}
    \item Qual è il massimo numero di archi in un grafo non orientato e bipartito con 15 vertici?
    \item Qual è il minimo numero di vertici in un grafo non orientato e bipartito, con lo stesso numero di vertici nelle due parti della bipartizione, e con 65 archi?
    \item Qual è il minimo numero di vertici in un grafo non orientato e bipartito, con lo stesso numero di vertici nelle due parti della bipartizione, e con 64 archi?
\end{itemize}
\textsc{Risposta:}

\paragraph{Es.9} Dimostrare che un grafo con n vertici e più di \( \frac{n^2}{4}\) archi non può essere bipartito.\\

\noindent
\textsc{Risposta:}

\paragraph{Es. 10} Stabilire se i due seguenti grafi sono isomorfi, cercando di costruire l'isomorfismo. \\
%Imm grafi

\noindent
\textsc{Risposta:} sono isomorfi e l'isomorfismo è il seguente: 
\begin{itemize}
    \item \( a \longleftrightarrow 3\)
    \item \(e \longleftrightarrow 5\)
    \item \( d \longleftrightarrow 4\)
    \item \( f \longleftrightarrow 2\)
    \item \(c \longleftrightarrow  1\)
    \item \(b \longleftrightarrow 6 \)
\end{itemize}

\paragraph{Es. 11} Dimostrare che tutti i grafi (semplici ) con 5 vertici e con ciascun vertice di grado 2 sono
isomorfi. La stessa proprietà vale anche per grafi con 6 vertici?\\

\noindent
\textsc{Risposta:} Tutti i grafi con 5 vertici, con ciascun vertice di grado 2, sono circuiti di lunghezza 5($C_5$) e quindi tutti isomorfi tra loro.

\paragraph{Es. 12} Tracciare una rappresentazione piana dei seguenti grafi, dove ogni arco è un segmento retto
(notare che "e" è uno arco qualsiasi).\\

\noindent
\textsc{Risposta:}

\paragraph{Es. 13} Quale dei seguenti grafi è planare?\\

\noindent
\textsc{Risposta:} 

\paragraph{Es. 14} Stabilire quanti vertici deve avere un grafo planare connesso con 5 facce e con 10 archi.\\
Disegnare un grafo con queste caratteristiche.\\

\noindent
\textsc{Risposta:} Planare e connesso $\Rightarrow$ vale la formula di Eulero $\Rightarrow r=e-v+2$ \\
\(\Leftrightarrow 5=10-v+2 \Leftrightarrow v=7\)

\paragraph{Es. 15}Stabilire se può esistere un grafo semplice non orientato con le caratteristiche indicate. In caso affermativo darne un esempio, in caso negativo spiegare perché non può esistere.
\begin{enumerate}
    \item Esiste un grafo planare semplice con 9 vertici e 22 archi?
    \item Esiste un grafo planare semplice con 9 vertici e 21 archi?
    \item Esiste un grafo planare semplice e bipartito con 8 vertici e 13 archi?
    \item Esiste un grafo planare semplice e bipartito con 8 vertici e 12 archi?
\end{enumerate}
\textsc{Risposta:}
\begin{enumerate}
    \item Se esiste, allora deve valere \(e \leq 3v-6 \Leftrightarrow 22 \leq 3.9-6=21\), quindi non esiste.
    \item La disequazione \(e \leq 3v -6\) questa volta da $21<21$ vero, ma non è sufficiente per affermare che il grafo esiste, devo trovare un esempio.
    %grafo
    \item Se esiste allora deve valere \(e \leq 2v -4 \Leftrightarrow 13 \leq 2*8 -4= 12 \) quindi non esiste.
    \item La disequazione dà $12 \leq 12$, devo cercare un esempio.
    %grafo
\end{enumerate}

\paragraph{Es. 16} Per quali valori di n il grafo $K_n$ è planare? Per quali valori di $r$ e di $s$ il grafo completo bipartito $K_r,s$ è planare?\\

\noindent
\textsc{Risposta:} $K_n$ è planare $\Leftrightarrow n \leq 4$. \\
Dato che \(r=s=3 \) non è vero, e \(r \geq 3 , s \geq 3\) nemmeno, \(r=2\) e un $s$ qualsiasi si: è planare per $r\leq 2$ e $s$ qualsiasi o viceversa.

\paragraph{Es. 17} Considerare il grafo non orientato G(V,E) qui sotto disegnato e rispondere alle seguenti domande.
\begin{enumerate}
    \item Il grafo G è bipartito?
    \item Calcolare il grado minimo, il grado massimo e il grado medio dei vertici
    \item Percorrendo i vertici del grafo in ordine alfabetico, trovate un circuito Hamiltoniano; usatelo per
stabilire se il grafo G è planare. Se la risposta è affermativa, disegnate il grafo in modo piano; se
invece la risposta è negativa, fornite $K\ped{3,3}$ o $K_5$ come minore
    \item Determinare K(G) e Ke(G)
\end{enumerate}
\textsc{Risposta:} 

\paragraph{Es. 18} È possibile disegnare un grafo planare semplice con 4 vertici e 4 facce?
Se possibile disegnarlo.\\

\noindent
\textsc{Risposta:}

\paragraph{Es. 19} È possibile disegnare un grafo planare semplice con 5 vertici e 6 facce?
Se possibile disegnarlo.\\

\noindent
\textsc{Risposta:}

\paragraph{Es. 20} È possibile disegnare un grafo planare semplice con 11 vertici in cui ogni vertice abbia grado
maggiore o uguale a 5? \\

\noindent
\textsc{Risposta:}


\newpage