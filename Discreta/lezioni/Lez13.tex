\section{Tredicesima lezione: Relazioni di ricorrenza}

Data una procedura con n oggetti, si vuole contare il numero di modi di eseguirla.\\
$a_n =$ numero di modi di eseguire la procedura con n oggetti ($n \geq 0$)\\
$a_0, a_1, a_2, \dots a_n, \dots $: successione di numeri da determinare

\subsection{Relazioni di ricorrenza} 
formula ricorsiva che esprime $a_n$ in funzione dei
precedenti termini della successione: \\
Soluzione di una relazione di ricorrenza %uguale
 formula esplicita per $a_n$ , che dipende solo
da n.

\subsection{Esempi}


\subsection{Dividi e conquista}

\subsection{Relazioni lineari omogenee}

%esempii

\subsection{Relazioni non omogenee}

%esempi







\newpage