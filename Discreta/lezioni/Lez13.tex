\section{Tredicesima lezione: Relazioni di ricorrenza}

Data una procedura con n oggetti, si vuole contare il numero di modi di eseguirla.\\
$a_n =$ numero di modi di eseguire la procedura con n oggetti ($n \geq 0$)\\
$a_0, a_1, a_2, \dots a_n, \dots $: successione di numeri da determinare.

\subsection{Relazioni di ricorrenza} 
La relazione di ricorrenza è una formula ricorsiva che esprime $a_n$ in funzione dei
precedenti termini della successione:
\[ \left\{ 
    \begin{array}{rcl} 
    a_n = f(a_0, a_1, a_2, \dots , a\ped{n-1}, n) \\
    a_0= \dots \\
    a_1 = \dots 
    \end{array} \right. \]    

In cui $a_0$ e $a_1$ sono condizioni iniziali.   \\
Soluzione di una relazione di ricorrenza $=$ formula esplicita per $a_n$ , che dipende solo da n.

\subsection{Esempi}

\paragraph{Es. 1} Permutazioni di n oggetti distinti

\paragraph{Es. 2} Contare il numero di modi di salire una scala con n gradini, se siamo in grado di fare passi da
uno o due gradini: \textit{relazione di Fibonacci}

\paragraph{Es. 3} Calcolare il numero di confronti necessari per trovare l'elemento
massimo di una sequenza di lunghezza n (supponendo che n sia una potenza
di 2).

\paragraph{Es. 4} Torre di Hanoi\\
\textit{Problema:} spostare la torre di dischi dal piolo A al piolo C.\\
\textit{Regole:}
\begin{itemize}
    \item si può spostare un solo disco alla volta;
    \item si può muovere un disco solo se è il primo della sua pila (quello più in alto);
    \item un disco spostato può essere appoggiato solo su un piolo vuoto o sopra un disco di
raggio maggiore.
\end{itemize}

\paragraph{Es. 5} Evoluzione di un capitale investito in banca con tasso di interesse 5\%.

\paragraph{Es. 6} I modi di selezionare k membri di un comitato presi fra n persone sono i modi di
selezionare il comitato in cui il sig. Rossi non è presente (k persone prese fra $n-1) + i$ modi di
selezionare il comitato in cui il sig. Rossi E' presente ($k-1$ persone prese fra $n-1$).
(Alla base della costruzione del triangolo di Pascal).

\subsection{Dividi e conquista}
Ordinamento di una lista di n numeri, o algoritmo di merge sort: data una serie di numeri, li divido a metà in modo ricorrente e ordino ogni parte divisa per poi riunirla alle altre.\\ 
\textit{Esempio:} \\
1 9 5 2    3 8 7 4 \\
$\Downarrow$  $\Downarrow$\\
1 2 5 9    3 4 7 8 \\
$\Downarrow$  $\Downarrow$\\
1 2 3 4 5 7 8 9 \\


\subsection{Relazioni lineari omogenee}

%esempii

\subsection{Relazioni non omogenee}

%esempi







\newpage