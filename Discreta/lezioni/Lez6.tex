\section{Sesta lezione: Grafi planari}
Un grafo non orientato si dice \emph{planare} se lo si può disegnare sul piano senza intersezioni tra gli archi. \\
Tale disegno viene detto \emph{rappresentazione piana} del grafo.\\

\noindent
Come stabilire se un grafo è planare? \\
Problema polinomiale, anche se l'algoritmo è complicato. Ne daremo una versione semplificata, chiamata il metodo del cerchio e delle corde.\\

\subsection{Minori}
Contrazione di un arco u,v. Identifico u, v in un unico vertice. Arco u,v è diventato
un cappio ( e viene usualmente rimosso.) \\
G'(V',E') è minore di G(V,E) se G' è ottenuto da G tramite:
\begin{itemize}
\item contrazione di archi
\item rimozione di archi
\item rimozione di vertici isolati
\end{itemize}
\textbf{Teorema:} G planare, G' minore di G, allora G' planare. \\
Ognuna delle 3 operazioni applicata ad una rappresentazione piana di G, mantiene piana la rappresentazione.

\subsection{Teorema di Kuratowski}
Un grafo è planare se e solo se non contiene $K\ped{3,3}$ o $K_5$ come minore.\\

\noindent
\subsection{Disegni equivalenti per grafi planari:} Dato un grafo planare, ci sono molte sue rappresentazioni piane, tutte equivalenti. \\
Ogni rappresentazione piana divide il piano in regioni o facce. Indichiamo con r il numero di queste
regioni. Il numero r dipende solo dal grafo, non dalla particolare rappresentazione piana disegnata. 

\subsubsection{Metodo del cerchio e delle corde}
Dato un grafo G(V,E) non orientato:
\begin{enumerate}
\item trovare, se esiste, un circuito di lunghezza massima ( che
contenga tutti i vertici del grafo, lo chiameremo circuito
Hamiltoniano)
\item disegnare questo circuito come un grande cerchio
\item scrivere l'elenco degli archi del grafo che non sono
contenuti nel circuito; li chiamiamo corde e li inseriamo
internamente o esternamente al cerchio, cercando di
evitare gli incroci, seguendo i passi 4 e 5
\item scegliamo una corda e inseriamola, ad esempio,
internamente al cerchio, togliendola dall'elenco;
\item Tutte le corde che la incrociano debbono essere inserite
esternamente. Se la rappresentazione e' piana,
proseguiamo. Altrimenti il grafo non e' planare.
\end{enumerate}
\emph{Note:}
\begin{itemize}
\item Attenzione all'ordine con il quale si inseriscono le corde:
non possiamo fare scelte, solo la prima corda può essere
posta internamente o esternamente a nostra scelta.
\item Questa procedura non è sempre applicabile; in
particolare, non è applicabile se il grafo non contiene un
circuito hamiltoniano.
\end{itemize}

\subsubsection{Esempi}

\subsection{Formula di Eulero}

\subsection{Politopi convessi}

\subsubsection{Corollari}

\subsection{Problema dei 4 colori}

\subsubsection{Esercizi}

\newpage
