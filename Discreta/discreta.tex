\documentclass[12pt,a4paper]{article}

\usepackage[english,italian]{babel}
\usepackage[utf8]{inputenc}
\usepackage[T1]{fontenc}

\usepackage{hyperref}

\usepackage{graphicx} %img/media
\usepackage{enumitem} %lists
\usepackage{listings} %coding environment 
\usepackage{mathtools} %math package
\usepackage{amssymb}


\begin{document}
    \author{Sara Righetto}
    \title{Appunti di Matematica Discreta}

    \maketitle

    \newpage
    \tableofcontents

    \newpage

    \begin{abstract}
        La matematica discreta, è lo studio di strutture matematiche che sono fondamentalmente discrete, nel senso che non
    richiedono il concetto di continuità che voi vedete nei corsi di analisi. \par
    La maggior parte degli oggetti studiati nella matematica discreta sono insiemi numerabili come i numeri interi o
    insiemi finiti, come gli interi da 1 a k. \par
    Altri oggetti importanti sono i grafi, che modellano reti di connessione. \\

        La matematica discreta è diventata famosa per le sue applicazioni in informatica. I concetti e le notazioni della matematica discreta sono utili per lo studio o la modellazione di oggetti o problemi negli algoritmi informatici e nei linguaggi di programmazione.
    \end{abstract}

\begin{center}
    \textit{Argomenti:}
    \begin{enumerate}
        \item Teoria dei grafi 
        \item Metodi di conteggio
        \item Relazioni di ricorrenza
    \end{enumerate}
\end{center}

    \newpage


%PRIMA LEZIONE
    \section{Prima lezione: Introduzione}

    \subsection{Teoria dei grafi}
        Grafo = G(V,E) \par
        \textbf{Planarità di un circuito stampato:} 
Nei circuiti stampati, le componenti elettroniche sono collegate da piste conduttrici stampate su una tavoletta
isolante; queste piste non si possono incrociare.\\

\textsc{Problema del postino cinese} \\
Un postino deve consegnare la posta nelle seguenti strade. Può partire da A e tornare in A senza percorrere due volte
la medesima strada ? \par
Equivalente: Disegnare la seguente figura senza sollevare la penna dal foglio (partendo e tornando in un punto
arbitrario) \\

\textsc{Spelling checker} \\
Cercare una parola nel dizionario. Zucca: si ricorre all'uso degli alberi binari \\

\textsc{Torneo ad eliminazione} \\
16 Giocatori: Torneo ad eliminazione diretta. Anche in questo caso si usano gli alberi binari.

\subsection{Metodi di conteggio} 
Contare un numero (finito) di oggetti non è sempre facile ma è un problema che si pone spesso in
informatica: Per esempio contare le operazioni fatte da un algoritmo per misurarne l'efficenza. \\

\textsc{Metodi di conteggio = Enumerazione = Calcolo combinatorio}
 \\

\begin{itemize}
        \item Numero di parole con un fissato alfabeto
\item Numero di sequenze binarie, numero di sottoinsiemi di n elementi
\item Numero di soluzioni intere di una data equazione lineare
\item Numero di targhe automobilistiche
\item Probabilità discreta: Quale è la probabilità che 4 carte diano un poker? (d'assi?)
\end{itemize}

\subsection{Relazioni di ricorrenza}
    Spesso è difficile calcolare direttamente una quantità (come il numero K\ped{n}
di sottoinsiemi di n
elementi con una data proprietà). E' più semplice trovare una relazione di ricorrenza che
determina \textit{K\ped{n} in funzione di K\ped{n-1}}. \\

\textsc{Relazioni di ricorrenza = Equazioni alle differenze finite}
 \\

Problemi ricorsivi:
\begin{enumerate}
\item calcolo del determinante di una matrice quadrata
\item numeri di Fibonacci
\item calcolo di n!
\item Torre di Hanoi
\end{enumerate} 

\par
\texttt{Esempio 1.1:} \par
    \textit{a}\ped{n} = \textit{a}\ped{n-1} * n \rightarrow \textsc{relazione di ricorrenza} \par
    \textit{a}\ped{0} = 1 \rightarrow \textsc{condizione iniziale} \\

    \textit{a}\ped{n} = n!  \rightarrow \textsc{soluzione} \\

\par
\texttt{Esempio 1.2:} \par
\textbf{Evoluzione di un capitale investito in banca:}  All'anno 0 investo 1000EUR. Il mio investimento da un interesse del 4 \% annuo. Dopo n anni quanto e' il mio capitale? \\

\textit{a}\ped{0} = 1000 \par
\textit{a}\ped{1} = 1000 + $\frac{4}{100}$ * 1000 = 1.04 * \textit{a}\ped{0} \par
\textit{a}\ped{n} = (1.04) * \textit{a}\ped{n-1} \rightarrow \textsc{relazione di ricorrenza} \par

\textit{a}\ped{2} = 1.04 * \textit{a}\ped{1} = (1.04)^2 * \textit{a}\ped{0} \par

\textit{a}\ped{n} = (1.04)^n * \textit{a}\ped{0} \rightarrow \textsc{soluzione} 
\\

\texttt{Esempio 2:} \par
\textbf{Calcolare il massimo di una stringa ($x_1$, .. , $x_n$ ) di n numeri} \par

$a_n$ = numero di confronti necessari per calcolare il massimo di una stringa di lunghezza n \\

\textit{a}\ped{n} = \textit{a}\ped{n/2} + \textit{a}\ped{n/2} +1 \rightarrow \textsc{relazione di ricorrenza} \par

\textit{a}\ped{2} = 1 \rightarrow \textsc{soluzione} \par 

\newpage


%SECONDA LEZIONE

\section{Seconda lezione: i grafi}
\subsection{Grafi non orientati}
    \textsc{definizione:} un grafo (non orientato) è una coppia G(V,E), dove:
    \begin{itemize}
\item V={v1, ...,vn} è un insieme finito di vertici (nodi)
\item E è un insieme di archi che sono coppie non ordinate i vertici
    \end{itemize}

\textit{ex:} \par
V = \{ v\ped{1}, .. , v\ped{8} \} \par
E = \{ (v\ped{1}, v\ped{3}), (v\ped{1}, v\ped{7}), (v\ped{2}, v\ped{3}), .. \} \\

Definizioni:
\begin{itemize}
    \item estremi di un arco
    \item arco incidente nei suoi estremi
    \item vertici adiacenti (nonadiacenti) : sono gli estremi di un arco
    \item grado d(v) del vertice v: Numero di volte in cui v è estremo di un arco. d(v3)=3, d(v8)=0
\end{itemize}

\paragraph{Due proprietà fondamentali} \par
    Per ogni grafo G(V, E) vale la seguente relazione: \par

\begin{center}
    \sum \ped{v \in V} d(v) = 2 \mid E \mid \\

\end{center}

    Poichè ogni arco ha 2 estremi, contribuisce 2 alla somma dei gradi. \\
    \textsc{CONSEGUENZA:} Ogni grafo ha un numero pari di vertici di grado dispari. \\
    Ogni grafo semplice ha al massimo \left( \begin{array}{c} n \\ 2 \end{array} \right) = $\frac{n(n-1)}{2}$ archi dove n= \mid V \mid. \par
    \left( \begin{array}{c} n \\ 2 \end{array} \right) è il numero di coppie non ordinate di n elementi.

\subsubsection{Terminologia}

\textbf{Cammino:} sequenza di vertici distinti, dove ogni
coppia di vertici consecutivi nel cammino è
collegata da un arco. \par
\textbf{Circuito o ciclo:} cammino nel quale il primo e
l'ultimo vertice sono coincidenti \par
\textbf{Lunghezza:} Numero di archi.\par
\textbf{Pari, dispari:} Parita' della lunghezza. \par
\textbf{Connesso:} per ogni coppia di vertici esiste un
cammino che li collega \par
\textbf{Percorso:} sequenza di vertici \textsc{NON NECESSARIAMENTE} distinti, dove ogni coppia di vertici
consecutivi nel percorso è collegata da un arco. (Cioe' e' una maniera di attraversare parte del
grafo, possibilmente passando piu' di una volta per lo stesso vertice o arco). Un percorso e' chiuso
se le estremita' coincidono. \\

\textbf{Teorema:} Dato un percorso con estremita' v\ped{1}, v\ped{n} esiste un cammino con estremita' v\ped{1}, v\ped{n} \par
\textbf{Teorema:} Un percorso chiuso di lunghezza dispari contiene un ciclo di lunghezza dispari. \\
Se gli unici vertici coincidenti sono le estremita' allora il percorso e' un ciclo dispari. Altrimenti il percorso si partiziona in due percorsi di lunghezza minore, uno pari e l'altro dispari. \par

\subsubsection{Alcuni grafi} \par
\textbf{Grafo completo:} ogni coppia di vertici è adiacente. E' denominato $K_n$ (n è il numero di vertici) \par
\textbf{Bipartito:} i vertici sono partizionabili in due sottoinsiemi, U e W, e ogni arco è incidente in un vertice di U e uno di W. \par
\textbf{Grafo bipartito completo $K_n1,n2$} Grafo bipartito, le due parti della bipartizione hanno \ped{n1}, \ped{n2} vertici rispettivamente ed ogni vertice di una parte è adiacente a tutti i vertici della seconda parte. \par
\textbf{Foresta:} Grafo senza cicli (=aciclico) \par
\textbf{Albero:} Foresta connessa. \par
Un grafo è K-regolare se ogni nodo ha grado k. I grafi 3-regolari si chiamano \textsc{grafi cubici}. \\


\textbf{Disegnare un grafo} \par
 Un "disegno" di un grafo ne illustra le proprietà. Ma i "disegni" dello stesso grafo sono molteplici. Noi
diremo che due grafi sono \textsc{ISOMORFI} se uno è il disegno dell'altro. \\
Formalmente, G(V,E) e G’(V’, E’) sono isomorfi se esiste una biezione F: che porta V in V’ e che preserva le
adiacenze. Cioè uv è un arco di G se e solo se f(u)f(v) è un arco di G’. \\

\textbf{Multigrafi} Finora abbiamo visto grafi SEMPLICI, perche' fra ogni coppia di vertici c'e' al massimo un arco. In generale possono esserci ARCHI PARALLELI, cioe' archi che hanno le stesse estremita' e CAPPI: Archi
con le estremita' coincidenti. Tali grafi si dicono MULTIGRAFI. \par
\textit{GRADO DI UN NODO:} I cappi contribuiscono 2 al grado dell'unica estremità.\\

\textbf{Sottografi:} Un sottografo di un grafo G(V,E) è un grafo G'(V', E') con V'\subseteq V e E'\subseteq E. \par
Dato G(V,E) un suo sottografo G'(V',E') si dice \textsc{indotto} da V' se E' è il sottoinsieme di archi in E con entrambe le estremità in V'. Cioè per ogni arco (u,v) \in E, se u \in V' e v \in V' allora (u,v) \in E'. \\


\subsection{Grafi orientati}
\textsc{Definizione:} un grafo orientato è una coppia D(V,A), dove: \par
V={v\ped{1}, ...,v\ped{n}} è un insieme finito di nodi o vertici. L'insieme A di archi è un insieme di coppie ORDINATE di vertici. Intuitivamente. Un arco ha un senso di percorrenza, dalla testa alla coda. \\

Nodo iniziale e finale di uno arco = coda e testa dell'arco\\
\begin{itemize}
\item \textbf{Archi paralleli:} hanno la stessa testa e la stessa coda.
\item \textbf{Cappio:}la testa e la coda coincidono.
\end{itemize}
\textbf{Grafo orientato Semplice:} non ci sono cappi e nemmeno archi paralleli.
\textbf{Multigrafo orientato:} ci sono cappi e archi paralleli. \\

\subsubsection{Terminologia}
\textbf{Cammino orientato:} sequenza di nodi distinti, dove ogni coppia di nodi consecutivi nel cammino è
collegata da un arco (nodo iniziale, nodo finale) \par
\textbf{Lunghezza del cammino:} numero di archi nel cammino. \\

Un grafo orientato e' \textit{Fortemente Connesso:} per ogni coppia di nodi u, v esiste un cammino orientato da u
a v ed un cammino orientato da v a u. \\

\textbf{Circuito:} cammino nel quale il primo nodo coincide con l'ultimo. \par
\textbf{Lunghezza del circuito:} numero di archi. \par
\textbf{Parità del circuito:} parità della lunghezza. \\

\subsubsection{Proprietà dei grafi orientati}
\textbf{\textit{Semplice} massimo numero di archi}: numero di coppie ORDINATE di n=\mid V \mid elementi. \par
\textbf{In-degree(v)} = Grado entrante di un nodo v: numero di archi entranti in v \Rightarrow d$^in$(v) \par
\textbf{Out-degree(v)} = Grado uscente di un nodo v: numero di archi uscenti da v \Rightarrow d$^out$(v) \\

Per ogni grafo orientato D(V,A) vale: \mid A \mid = \sum $_v \in V$ d$^in$(v) = \sum $_v \in V$ d$^out$(v). \\

\paragraph{Teorema} \\

In ogni grafo orientato D(V,A), sono uguali tra loro:
\begin{itemize}
\item la somma dei gradi uscenti dei nodi
\item la somma dei gradi entranti dei nodi
\item il numero di archi del grafo
\end{itemize}

\begin{center}
\sum\ped{v \in V} d$^out$(v) = \sum\ped{v \in V} d$^in$(v) = \mid A \mid \\

\end{center}

Un arco uv un contribuisce 1 a d$^out$(v) e 1 a d$^in$(v)
\\

\paragraph{Tornei:} \\

Un \textit{TORNEO} è un grafo orientato in cui per ogni coppia di vertici esiste \textsc{ESATTEMENTE} uno degli archi uv , uv. \par
(Si chiamano tornei (tournaments) perche indicano gli esiti di un torneo e il senso di ogni arco indica l'esito dell'incontro fra i giocatori rappresentati dai suoi estremi).

\subsubsection{Rappresentazione di grafi mediante matrici}
\begin{enumerate}
\item \textsc{Matrice di incidenza vertici/archi}
\item \textsc{Matrice di incidenza nodi/archi}
\item \textsc{Matrice di adiacenza vertici/vertici o nodi/nodi}
\item List of edges, linked adjacency list, forward star, backward star ect..
\end{enumerate}


\textbf{}

\end{document}